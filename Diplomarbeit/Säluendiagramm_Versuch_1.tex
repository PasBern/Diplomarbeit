\documentclass[
10pt,			%Schriftgr��e
paper=a4,		%Papierformat
ngerman,		%Deutsche Silbentrennung
BCOR=0pt,		%Bindekorrektureinstellung=AN (15mm) Bindekorrektureinstellung=AUS > 0pt
%twoside,
DIV=calc,		%Satzspiegelberechnung
headinclude,	%Kopfzeile bei Satzspiegelberechnung mit einbeziehen
headsepline,	%Trennlinie unter der Kopfzeile
numbers=noenddot,
]{scrreprt}
\usepackage[T1]{fontenc}
\usepackage{lmodern}
\usepackage[sc]{mathpazo}

% Satzspiegel wird neu berechnet
\typearea[current]{calc}

\usepackage[ansinew]{inputenc}

% Paket f�r besseres Schriftbild
\usepackage{microtype}


% Allgemeine Pakete:

\usepackage{xcolor}
\usepackage[ngerman]{babel}
\usepackage{relsize}
\usepackage{paralist}
\usepackage{mdwlist}
\usepackage{typearea}
\usepackage{setspace}
%\usepackage{textcomp}
\usepackage{marvosym}

\usepackage{fancybox}
\usepackage{array}
\usepackage{amsfonts,amsmath,amssymb}
\usepackage{pifont}
\usepackage{calc}
\usepackage{ifthen}
\usepackage{tikz}
\usepackage{pgf-pie}
\usepackage{pgfplots}
\pgfplotsset{compat=1.8}


\usetikzlibrary{shapes,arrows,shadows}

\usepackage{scrpage2}

% Pakete f�r Abbildungen
\usepackage{graphicx}
\usepackage{floatrow}
\usepackage{floatpag}



% Fu�notengefrickel muss vor HYPERREF geladen werden
\usepackage[bottom,flushmargin,hang,multiple]{footmisc}

%Hyperxmp-Paket einbinden f�r PDF-Metadaten
\usepackage{hyperxmp} 

% Rahmen um Hyperlinks unterbinden
\usepackage[bookmarks=true,pdftoolbar=true,pdfmenubar=true,pdfstartview={FitH},citecolor=magenta,hidelinks,breaklinks=true]{hyperref}

\hypersetup{ %
pdftitle={Gasmarktliberalisierung im Baltikum}
pdfauthor={Pascal Bernhard}
pdfproducer={Texlive-LaTeX 2013 on Sabayon GNU/Linux 3.12 & 3.13}
pdfcopyright={Creative Commons 3.0 CC-BY}
pdflicenseurl={https://creativecommons.org/licenses/by/3.0/de/legalcode}
pdfmetalang={de-DE}
}




%%% Auf tats�chlichen Namen des Abschnitts verlinken
\usepackage{titleref}
\usepackage{nameref}



% Abst�nde zwischen �berschrift und Textk�rper modifizieren [PAKET NICHT INSTALLIERT!!!!]
%\titlesec

% Gleiche Schriftart f�r Hyperlinks
\urlstyle{same}


%  Gefrickel um URL-Links vern�nftig umzubrechen
\makeatletter
\g@addto@macro\UrlBreaks{
  \do\a\do\b\do\c\do\d\do\e\do\f\do\g\do\h\do\i\do\j
  \do\k\do\l\do\m\do\n\do\o\do\p\do\q\do\r\do\s\do\t
  \do\u\do\v\do\w\do\x\do\y\do\z\do\&\do\1\do\2\do\3
  \do\4\do\5\do\6\do\7\do\8\do\9\do\0}
% \def\do@url@hyp{\do\-}

% Hiermit soll einer �bervolle Box verhindert werden -- funktioniert sogar irgendwie
\g@addto@macro\UrlSpecials{\do\/{\mbox{\UrlFont/}\hskip 0pt plus 1pt}}
\makeatother

% Hier kommen Optionen der KOMA-Klasse zu �berschriften
\KOMAoptions{headings=small}

% Einstellungen fpr Kopfzeilen
\ihead{\leftmark} 
\ohead{\rightmark} 
\chead{} 
\pagestyle{scrheadings} 
\automark[section]{chapter}

\setfootnoterule[0.8pt]{10cm} % H�he und L�nge des Trennstriches f�r Fu�noten anpassen
\setkomafont{caption}{\sffamily} % Bezeichnungen von Bilder werden serifenlos und fett gesetzt
\setkomafont{captionlabel}{\sffamily\bfseries} % Beschreibungen von Bilder werden serifenlos und fett gesetzt

\renewcommand{\headfont}{\small\sffamily\mdseries} % Schriftart der Kopfzeile �ndern



%%%----------------------------------------------------------------------
\definecolor{dunkelgrau}{gray}{0.20}
\definecolor{hellgrau}{gray}{0.40}
\definecolor{MidnightBlue}{RGB}{0,103,149}
\definecolor{NavyBlue}{RGB}{0,110,184}
\definecolor{MidBlue}{rgb}{0.173,0.212,0.597}
\definecolor{Dschungel}{cmyk}{0.99,0,0.52,0.2}


%%% Palatino ben�rtigt gr��eren Raum zwischen den Zeilen
\linespread{1.05}


\onehalfspacing
\typearea[current]{calc}



%%% Schriften umdefinieren:

\newcommand{\changefont}[3]{
\fontfamily{#1} \fontseries{#2} \fontshape{#3} \selectfont}

\newcommand{\textemph}{\textsl{\textbf}}
%%%----------------------------------------------------------------------
\begin{document}

\begin{spacing}{1}

%%% Deckblatt Start
\thispagestyle{empty}


\begin{center}
\Large{Freie Universit\"{a}t Berlin}\\
\end{center}
 
 
\begin{center}
\Large{Fachbereich Politikwissenschaft}
\end{center}
\begin{verbatim}
 
 
\end{verbatim}
\begin{center}
\textbf{\LARGE{\changefont{ppl}{b}{n}
\sffamily{Diplomarbeit}}}
\end{center}
\begin{verbatim}
 
 
\end{verbatim}
\begin{center}
\textbf{\changefont{ppl}{b}{n}
\sffamily{im Studiengang Politik}}
\end{center}
\begin{verbatim}
 
 
\end{verbatim}
 
\begin{flushleft}
\begin{tabular}{lll}

\textbf{Thema:} & & Energiepolitik im Baltikum --\\
& & Umsetzung des 3. EU-Liberalisierungspakets zum\\
& & Gasbinnenmarkt in Litauen und Lettland\\
& & \\
& & \\
& & \\



\textbf{eingereicht von:} & & Pascal Bernhard \flq{}pascal.bernhard@belug.de\frq{}\\
& & \\
& & \\
\textbf{eingereicht am:} & & 11. April 2014\\
& & \\
& & \\
\textbf{Betreuer:} & & Herr Prof. Dr. Manfred Kerner\\
\textbf{Betreuer:} & & Frau Prof. Dr. Miranda Schreurs
\end{tabular}

\newpage

\thispagestyle{empty}


\textbf{Diese Diplomarbeit widme ich Frau Andrea Volmary und Herrn Kai-Uwe Christoph, ohne deren Hilfe mein Studienabschluss nicht m�glich gewesen w�re.}

\end{flushleft}

\end{spacing}
 
%Deckblatt Ende


%%%Palatino als Standardschrift
\renewcommand{\rmdefault}{ppl}

\renewcommand\thesubsection{\thesection.\alph{subsection}}

%%%Inhaltsverzeichnis----------------------------------------------------------------------

\cleardoublepage
\begingroup
\renewcommand*{\chapterpagestyle}{empty}
\pagestyle{empty}
\tableofcontents
\clearpage
\endgroup

%%%----------------------------------------------------------------------
\setlength{\parindent}{30pt}
\setlength{\parskip}{0pt}


\definecolor{myblue}{HTML}{92dcec}

\begin{center}


\begin{tikzpicture}
 
  \draw (0cm,0cm) -- (15.5cm,0cm);  %Abzisse
  \draw (0cm,0cm) -- (0cm,-0.1cm);  %linkes Ende der Abzisse
  \draw (15.5cm,0cm) -- (15.5cm,-0.1cm);  %rechtes Ende der Abzisse
  
  \draw (-0.1cm,0cm) -- (-0.1cm,4.5cm);  %Ordinate
  \draw (-0.1cm,0cm) -- (-0.2cm,0cm);  %unteres Ende der Ordinate
  \draw (-0.1cm,4.5cm) -- (-0.2cm,4.5cm) node [left] {\%};  %oberes Ende der Ordinate

  \foreach \x in {1,...,4}  %Hilfslinien
    \draw[gray!50, text=black] (-0.2 cm,\x cm) -- (15.5 cm,\x cm) 
      node at (-0.5 cm,\x cm) {\x};  %Beschriftung der Hilfslinien

    \node at (7.5cm,5cm) {Wachstumsrate des realen BIP f�r die zehn
                           Bev�lkerungsreichsten Staaten der EU 2005};  %�berschrift

  \foreach \x/\y/\country in {0.5/4.1/Rum�nien,  %\x ist Anfang der S�ulen
                              2/3.7/Griechenland,  %\y ist H�he der S�ulen
                              3.5/3.5/Spanien,
                              5/3.5/Polen,
                              6.5/1.9/Gro�britannien,
                              8/1.5/Niederlande,
                              9.5/1.2/Frankreich,
                              11/0.9/Deutschland,
                              12.5/0.5/Portugal,
                              14/0.1/Italien}
    {
     \draw[fill=myblue] (\x cm,0cm) rectangle (0.5cm+\x cm,\y cm) %die S�ulen
       node at (0.5cm + \x cm,\y cm + 0.3cm) {\y}; %die Prozente �ber den S�ulen
     \node[rotate=45, left] at (0.6 cm +\x cm,-0.1cm) {\country}; %S�ulenbeschriftung
    };

\end{tikzpicture}


\end{center}


\end{document}