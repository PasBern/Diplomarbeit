% !TeX encoding = UTF-8


\documentclass[
10pt,			%Schriftgröße
paper=a4,		%Papierformat
ngerman,		%Deutsche Silbentrennung
BCOR=0pt,		%Bindekorrektureinstellung=AN (15mm) Bindekorrektureinstellung=AUS > 0pt
%twoside,		%Bindekorrektur für zweiseitigen Druck
DIV=calc,		%Satzspiegelberechnung
headinclude,	%Kopfzeile bei Satzspiegelberechnung mit einbeziehen
headsepline,	%Trennlinie unter der Kopfzeile
numbers=noenddot,
twocolumn,
]{article} 


\usepackage[T1]{fontenc}
\usepackage{lmodern}
\usepackage[sc]{mathpazo}


\usepackage[utf8]{inputenc}

% Paket für ein besseres Schriftbild
\usepackage{microtype}

% Allgemeine Pakete:

\usepackage{xcolor}
\usepackage[ngerman]{babel}
\usepackage{relsize}
\usepackage{paralist}
\usepackage{mdwlist}
\usepackage{setspace}
\usepackage{marvosym}

\usepackage{fancybox}
\usepackage{array}
\usepackage{amsfonts,amsmath,amssymb}
\usepackage{pifont}
\usepackage{calc}
\usepackage{ifthen}
\usepackage{tikz}



% Fußnotengefrickel muss vor HYPERREF geladen werden
\usepackage[bottom,flushmargin,hang,multiple]{footmisc}

%Hyperxmp-Paket einbinden für PDF-Metadaten
\usepackage{hyperxmp} 

% Rahmen um Hyperlinks unterbinden
\usepackage[bookmarks=true,pdftoolbar=true,pdfmenubar=true,pdfstartview={FitH},citecolor=magenta,hidelinks,breaklinks=true]{hyperref}



%%% Auf tatsächlichen Namen des Abschnitts verlinken
\usepackage{titleref}
\usepackage{nameref}



% Gleiche Schriftart für Hyperlinks
\urlstyle{same}



%  Gefrickel um URL-Links vernünftig umzubrechen
\makeatletter
\g@addto@macro\UrlBreaks{
  \do\a\do\b\do\c\do\d\do\e\do\f\do\g\do\h\do\i\do\j
  \do\k\do\l\do\m\do\n\do\o\do\p\do\q\do\r\do\s\do\t
  \do\u\do\v\do\w\do\x\do\y\do\z\do\&\do\1\do\2\do\3
  \do\4\do\5\do\6\do\7\do\8\do\9\do\0}
% \def\do@url@hyp{\do\-}

% Hiermit soll einer übervolle Box verhindert werden -- funktioniert sogar irgendwie
\g@addto@macro\UrlSpecials{\do\/{\mbox{\UrlFont/}\hskip 0pt plus 1pt}}
\makeatother


%%%----------------------------------------------------------------------
% Farbdefinitionen
\definecolor{dunkelgrau}{gray}{0.20}
\definecolor{hellgrau}{gray}{0.40}
\definecolor{MidnightBlue}{RGB}{0,103,149}
\definecolor{NavyBlue}{RGB}{0,110,184}
\definecolor{MidBlue}{rgb}{0.173,0.212,0.597}
\definecolor{Dschungel}{cmyk}{0.99,0,0.52,0.2}



%------------------------------------------------------------------------- 
% take the % away on next line to produce the final camera-ready version 
\pagestyle{empty}

%------------------------------------------------------------------------- 
\begin{document}

\title{Ethnische Konflikte in Osteuropa}

\author{Pascal Bernhard\\
Freie Universität Berlin\\ OIntroductiontto-Suhr Institut für Politikwissenschaft}


\maketitle
\thispagestyle{empty}

\begin{abstract}
   The ABSTRACT is to be in fully-justified italicized text, at the top 
   of the left-hand column, below the author and affiliation 
   information. Use the word ``Abstract'' as the title, in 12-point 
   Times, boldface type, centered relative to the column, initially 
   capitalized. The abstract is to be in 10-point, single-spaced type. 
   The abstract may be up to 3 inches (7.62 cm) long. Leave two blank 
   lines after the Abstract, then begin the main text. 
\end{abstract}



%------------------------------------------------------------------------- 
\section{Introduction}

Please follow the steps outlined below when submitting your 
manuscript to the IEEE Computer Society Press. Note there have 
been some changes to the measurements from previous instructions. 

%------------------------------------------------------------------------- 
\section{Instructions}

Please read the following carefully.

%------------------------------------------------------------------------- 
\subsection{Language}

All manuscripts must be in English.

%------------------------------------------------------------------------- 
\subsection{Printing your paper}

Print your properly formatted text on high-quality, $8.5 \times 11$-inch 
white printer paper. A4 paper is also acceptable, but please leave the 
extra 0.5 inch (1.27 cm) at the BOTTOM of the page.

%------------------------------------------------------------------------- 
\subsection{Margins and page numbering}

All printed material, including text, illustrations, and charts, must be 
kept within a print area 6-7/8 inches (17.5 cm) wide by 8-7/8 inches 
(22.54 cm) high. Do not write or print anything outside the print area. 
Number your pages lightly, in pencil, on the upper right-hand corners of 
the BACKS of the pages (for example, 1/10, 2/10, or 1 of 10, 2 of 10, and 
so forth). Please do not write on the fronts of the pages, nor on the 
lower halves of the backs of the pages.


%------------------------------------------------------------------------ 
\subsection{Formatting your paper}

All text must be in a two-column format. The total allowable width of 
the text area is 6-7/8 inches (17.5 cm) wide by 8-7/8 inches (22.54 cm) 
high. Columns are to be 3-1/4 inches (8.25 cm) wide, with a 5/16 inch 
(0.8 cm) space between them. The main title (on the first page) should 
begin 1.0 inch (2.54 cm) from the top edge of the page. The second and 
following pages should begin 1.0 inch (2.54 cm) from the top edge. On 
all pages, the bottom margin should be 1-1/8 inches (2.86 cm) from the 
bottom edge of the page for $8.5 \times 11$-inch paper; for A4 paper, 
approximately 1-5/8 inches (4.13 cm) from the bottom edge of the page.

%------------------------------------------------------------------------- 
\subsection{Type-style and fonts}



%------------------------------------------------------------------------- 
\subsection{Footnotes}

Please use footnotes sparingly%
\footnote
   {%
     Or, better still, try to avoid footnotes altogether.  To help your 
     readers, avoid using footnotes altogether and include necessary 
     peripheral observations in the text (within parentheses, if you 
     prefer, as in this sentence).
   }
and place them at the bottom of the column on the page on which they are 
referenced. Use Times 8-point type, single-spaced.


%------------------------------------------------------------------------- 
\subsection{Quellenverzeichnis}

\begin{compactitem}



%\sffamily


\begin{small}


	\item [\Rectsteel] \textbf{Bernauer, Julian \& Daniel Bochsler (2011):} Electoral Entry And Success Of Ethnic Minority Parties In Central And Eastern Europe: A Hierarchical Selection Model, in: \textsl{Electoral Studies}, Vol.30, S.738-755.


	\item [\Rectsteel] \textbf{Guibernaut, Montserrat (2004):} Anthony D. Smith On Nations And National Identity: A Critical Assessment, in: \textsl{Nations and Nationalisms}, Vol.10 (1/2), S.125-141.

	\item [\Rectsteel] \textbf{Hale, Henry, E. (2008):} The Foundations Of Ethnic Politics: Separatism Of States And Nations In Eurasia And The World, New York, USA.


\end{small}



	\end{compactitem}


%------------------------------------------------------------------------- 

\end{document}

