\documentclass[9pt,a4paper,ngerman]{report}
\usepackage[T1]{fontenc}
\usepackage{parskip}
\pagestyle{empty}
\usepackage[ansinew]{inputenc}


\usepackage{xcolor}
\usepackage[ngerman]{babel}
\usepackage{relsize}
\usepackage{paralist}
\usepackage{typearea}
\usepackage{setspace}
\usepackage{textcomp}
\usepackage{layouts}
\usepackage{amsmath}
\usepackage{fancybox}

\definecolor{dunkelgrau.80}{gray}{0.20}
\definecolor{dunkelgrau.60}{gray}{0.40}
\definecolor{Dschungelgr\"{u}n}{cmyk}{0.99,0,0.52,0.2}
\definecolor{midblue}{rgb}{0.173,0.212,0.597}


\usepackage[top=10mm,bottom=10mm,marginparsep=8pt]{geometry}

\usepackage{hyperref}


%%%-----------------------------------------------------------------------
\begin{document}



 
 	\begin{flushright}
 			{\small \sffamily{\textcolor{dunkelgrau.60}{Fragestellung\\
 							\emph{Pascal Bernhard}\\
 							Diplomarbeit\\
 							\emph{Prof. Dr. Manfred Kerner}}}}
 	\end{flushright}						

	

	\begin{center}
			{\shadowbox{\sffamily{\textbf{ 
				Energie im Baltikum
			\texttwelveudash{}Mehr Versorgungssicherheit durch gemeinsame europ\"{a}ische Energiepolitik?}}}}
	\end{center} 	

	
 
		
		\begin{itemize}
		
			\item{\textcolor{midblue}{\sffamily{\textbf{\emph{Kann die EU-Energiepolitik den baltischen L�nder aus ihrer Abh�ngigkeit von russischen Gaslieferungen helfen?}}}}}
 
 			\begin{itemize}
 			
 				\item{Anwendung von IB-Theorien \& Integrationsans\"{a}tzen 
 							auf den europ��ischen Integrationsprozess}
 				\item{\textsl{aktuelle politische Relevanz}}
 				
 					\begin{itemize}
 					
 					\item{\textcolor{dunkelgrau.60}{Energieversorgungssicherheit 
 								als gesamteurop\"{a}ische Herausforderung}}
 					\item{\textcolor{dunkelgrau.60}{Gasstreitigkeiten zwischen 
 								Russland und der 
 								Ukraine zeigen, dass \hyperref[Bedarf]{Handlungsbedarf} besteht}}
 					\end{itemize}
 					
 			\end{itemize}
 
		\end{itemize} 
 
 		\begin{itemize}
 			
 			\item{\textcolor{midblue}{\textbf{\textsl{Theoretische Grundlagen}}}}
 				
 				\begin{itemize}
 					
 					\item{(liberaler) \ref{Bedarf} Intergouvernementalismus zur Erkl\"{a}rung der 
 								Integrationsentscheidungen in der EU}
 					\item{Mitgliedsl\"{a}nder sind die Herren der Vertr\"{a}ge
 								\texttwelveudash {} Opt-out--M\"{o}glichkeiten im EGV}
 					\item{Energiesektor ist eng mit dem Nationalstaat verbunden}
 						
 						\begin{itemize}
 							
 							\item{\textcolor{dunkelgrau.60}{Energie ist essentiell 
 											f\"{u}r	Wirtschaft und Gesellschaft eines Landes}}
 							\item{\textcolor{dunkelgrau.60}{historisch gesehen waren 
 										Energieunternehmen stets unter staatlicher 
 										Kontrolle}}
 										
 						\end{itemize}
 	
 	
 	\reversemarginpar
 						
 					\item{\textsl{Intergouvernementalismus} erkl\"{a}rt bisherige
 								Integrationsschritte nur unzul\"{a}nglich}
 					\item{aufgrund der spezifischen Charakteristika des 
 								Energiesektors 
 								\marginpar{\scriptsize{\textcolor{cyan}{\glqq 
 								Collective-Action\grqq {} \texttwelveudash {} nat\"{u}rliche 
 								Monopole}}} macht es f\"{u}r die EU-Mitglieder Sinn, 
 								energiepolitische Fragen supranational zu regeln}
 	
 	\normalmarginpar
 							
 							\begin{itemize}
 							
 								\item{\textcolor{dunkelgrau.60}{\"{A}berwindung des 
 											Kooperationsdilemmas (Abwehr einer 	
 											\textsl{divide-et-impera} \texttwelveudash {}  
 															Strategie der F\"{o}rderl\"{a}nder)}}
 											
 								\item{\textcolor{dunkelgrau.60}{Effizienz\"{u}berlegungen}}
 								
 							\end{itemize}
 						
 					\item{Institutionengef\"{u}ge der EU gibt jedoch auch anderen 
 								Akteuren ein Mitspracherecht}
 					\item{der Rational-Choice \texttwelveudash {} Ansatz bietet 
 								sich an f\"{u}r die Analyse der Entscheidungsprozesse 
 								(Principal-Agent \texttwelveudash {} Struktur, 
 								Verhandlungsmodelle)}
 		
 			\end{itemize}
 			
 		\end{itemize}
 		
 \begin{center}	
 		\textbf{\textsl{Die 
 		\marginpar{\scriptsize{Sollte die 
 						Fragestellung weiter eingeschr\"{a}nkt werden?}}} Energiepolitik 
 						auf europ\"{a}ischer Ebene kommt nicht durch 
 						intergouvernementale Entscheidungen der	EU-Mitglieder zu 
 						Stande, sondern entsteht durch Verhandlung zwischen den 
 						Institutionen Kommission, Rat und EP, deren 
 						Integrationspr\"{a}ferenzen ber\"{u}cksichtigt werden m\"{u}ssen}
 	\end{center}
 							
	\begin{itemize}
		
		
		\item{\textcolor{midblue}{\textbf{\textsl{abh\"{a}ngige Variable}}}}	
			
			\begin{itemize}
				\item{Gradder Integration in Bezug auf Energiepolitik}
				\item{\textsl{Operationalisierung (ordinales Messniveau					
							\texttwelveudash {} 7-stufige Skala):}}
							
					\begin{itemize}
						\item{\textcolor{dunkelgrau.60}{\"{U}bertragung von 
									Kompetenzen an europ\"{a}ische Organe}}
						\item{\textcolor{dunkelgrau.60}{Bewilligung von Mitteln zur 
									Fortentwicklung einer europ. Energiepolitik}}
						\item{\textcolor{dunkelgrau.60}{politische Beschl\"{u}sse zur 
									Koordination nationaler Politiken}}
					\end{itemize}
					
			\end{itemize}
			
		\item{\textcolor{midblue}{\textbf{\textsl{unabh\"{a}ngige Variablen\marginpar{\scriptsize{\textcolor{Dschungelgr\"{u}n}{Ist der Entscheidungsmechanismus als Variable zu sehen?}}}}}}}
		
			\begin{itemize}
				\item{Pr\"{a}ferenzen der Mitgliedsstaaten zu Integration}
				\item{Vorschl\"{a}ge (Pr\"{a}ferenzen) der Kommission}
				\item{Rolle des Europ\"{a}ischen Parlaments}
				\item{\textsl{Operationalisierung (ordinales Messniveau					
							\texttwelveudash {} 7-stufige Skala):}}


		\reversemarginpar
									
							\begin{itemize}
							
				\reversemarginpar
					
					\item{\textcolor{dunkelgrau.60}{\marginpar{\scriptsize{\textcolor{cyan}{informelle Entscheidungsfindungsprozesse}}}Gr\"{u}nbuch und White 
												Papers, Statements, Entw\"{u}rfe des DG TREN, Staff 
												Working Documents}}
												
			\normalmarginpar	
			
												\item{\textcolor{dunkelgrau.60}{ Stellungnahmen 
															des	Rates,  Protokolle (soweit 
															zug\"{a}nglich), COREPER und 															Energiekommittee\marginpar{\scriptsize{\textcolor{Dschungelgr\"{u}n}{Ist die \glqq Willigkeit\grqq {} zur Umsetzung der Richtlinien ein Indikator f\"{a}r Pr\"{a}ferenzen der Mitglieder?}}}}}				
								\item{\textcolor{dunkelgrau.60}{Fortschrittsberichte der 
											Kommission}}
								\item{\textcolor{dunkelgrau.60}{Position des EP}}
								
							\end{itemize}
							
				\end{itemize}
				
	\end{itemize} 							
 							
\end{document}
