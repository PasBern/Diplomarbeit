\documentclass[9pt,a4paper,ngerman]{scrreprt}
\usepackage[T1]{fontenc}
\usepackage{parskip}
\pagestyle{empty}
\usepackage[ansinew]{inputenc}


\usepackage{xcolor}
\usepackage[ngerman]{babel}
\usepackage{relsize}
\usepackage{paralist}
\usepackage{typearea}
\usepackage{setspace}
\usepackage{textcomp}
\usepackage{layouts}
\usepackage{amsmath}
\usepackage{fancybox}
\usepackage{marvosym}

\definecolor{dunkelgrau.80}{gray}{0.20}
\definecolor{dunkelgrau.60}{gray}{0.40}
\definecolor{Dschungelgr\"{u}n}{cmyk}{0.99,0,0.52,0.2}
\definecolor{midblue}{rgb}{0.173,0.212,0.597}


\usepackage[top=10mm,bottom=10mm,marginparsep=8pt]{geometry}

%%%Palatino als Standardschrift
\renewcommand{\rmdefault}{ppl}


%%%-----------------------------------------------------------------------
\begin{document}


 	\begin{flushright}
 			{\small \sffamily{\textcolor{dunkelgrau.60}{Thesenpapier f�r m�ndliche Pr�fung\\
 							\emph{Pascal Bernhard}\\
 							\emph{Prof. Dr. Manfred Kerner\\
 							Prof. Dr. Franz Walk}}}}
 	\end{flushright}


\vspace{0.8cm}

	\begin{center}
			{\shadowbox{\sffamily{\textbf{ 
				{\Large �berregionale Zusammenarbeit in Europa am Beispeil der Euro-Regionen}}}}}
	\end{center} 	

\vspace{1.5cm}


\begin{quote}
\textit{ %
\textsf{\textbf{These:}\\
		Der unterschiedliche Handlungsspielraum der lokalen Selbstverwaltung, welcher durch das jeweilige nationale politische System gew�hrt wird, bestimmt Umfang und Erfolg transnationaler Kooperation auf regionaler Ebene.}}
\end{quote}


\vspace{0.8cm}


\subsection*{Regionale Zusammenarbeit in Europa}

\begin{itemize}
	\item \textbf{Antriebskr�fte regionaler Kooperation unterhalb der inter-gouvernementalen Ebene}


	\begin{compactitem}
		\item \textsl{Subsidiarit�tsprinzip:} regionale Probleme k�nnen besser von lokalen Institutionen bew�ltigt werden
		\item EU-Politik war nicht auf die Bed�rfnisse der territorialen Gliederung der Mitgliedsstaaten zugeschnitten
		\item bisherige M�glichkeiten der Zusammenarbeit auf europ�ischer Ebene nicht ausreichend
		\item durch Grenzen hervorgerufene Hindernisse sollen �berwunden werden
	\end{compactitem}


	\item \textbf{Rechtliche Grundlagen der grenz�berschreitenden Zusammenarbeit}

	\begin{compactitem}
		\item \textsl{Europ�ische Rahmenkonvention zur grenz�berschreitenden Zusammenarbeit (Konvention von Madrid, 1980)} gibt ersten rechtlichen Rahmen
		\item Zusatzprotokoll von 1995 gab erstmals die M�glichkeit, mit Gebietsk�rperschaften von Nicht-EU-Mitgliedern zusammenzuarbeiten
		\item \textsl{Vertrag von Maastricht 1992} erhob das Subsidiarit�tsprinzip zur allgemeing�ltigen Rechtsgrundlage
		\item Finanzierungm�glichkeiten f�r Projekte durch Europ�ische Territorial Zusammenarbeit (ETZ) nach Art. 158 EGV im Rahmen der Koh�sions- \& Strukturpolitik
	\end{compactitem}


	\item \textbf{kontinuierliche Wirtschaftskrise der Ukraine}

	\begin{compactitem}
	\item Absatzm�rkte f�r ukrainische Wirtschaft brachen weg (Landwirtschaft, Bergbau, R�stungsg�ter)
	\item Strukturwandel der sowjetischen Wirtschaft wurde von der Politik nicht vorangetrieben
	\item Lebensstandard der russischen Minderheit basierte auf sowjetischem System und war durch Marktreformen bedroht
	\item als einzige Branche florierte die Schattenwirtschaft
	\item sinkender Lebensstandard f�r die Mehrheit (Entlassungen, Inflation) bei zeitgleicher Bereicherung der Oligarchen
	\end{compactitem}

\end{itemize}

\vspace{0.8cm}



\begin{quote}
\textit{ %
\textsf{\textbf{These II:}\\
		Der Status quo ante der Ukraine wird voraussichtlich nicht wieder herzustellen sein, da weder die russische Minderheit, noch Russland hieran ein Interesse haben und dem Westen f�r entsprechende Schritte der Wille und die Einigkeit zu gemeinsamen Handeln fehlen.}}
\end{quote}


\subsection*{Literatur}

\begin{compactitem}
	\item [\Rectsteel] \textbf{Aalto, Pami \& Kirsten Westphal (2008):} Introduction, in: \textsl{Aalto, Pami (Hrsg.)}: The EU-Russian 	Energy Dialogue: Europe's Future Energy Security, Aldershot, Hampshire, UK, S.1-22.


	\item [\Rectsteel] \textbf{Aalto, Pami \& Kirsten Westphal (2008):} Introduction, in: \textsl{Aalto, Pami (Hrsg.)}: The Land Between: Conflict in the East European Borderlands, Aldershot, Oxfordshire, UK.


\end{compactitem}


\end{document}