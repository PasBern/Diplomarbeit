\documentclass[11pt,a4paper,ngerman]{scrreprt}
\usepackage[T1]{fontenc}
\usepackage{parskip}
\pagestyle{empty}
\usepackage[ansinew]{inputenc}


\usepackage{xcolor}
\usepackage[ngerman]{babel}
\usepackage{relsize}
\usepackage{paralist}
\usepackage{typearea}
\usepackage{setspace}
\usepackage{textcomp}
\usepackage{layouts}
\usepackage{amsmath}
\usepackage{fancybox}
\usepackage{marvosym}
\usepackage{pifont}
\usepackage{framed}

\definecolor{dunkelgrau.80}{gray}{0.20}
\definecolor{dunkelgrau.60}{gray}{0.40}
\definecolor{Dschungelgr\"{u}n}{cmyk}{0.99,0,0.52,0.2}
\definecolor{midblue}{rgb}{0.173,0.212,0.597}




\usepackage[top=10mm,bottom=10mm,marginparsep=8pt]{geometry}

%%%Palatino als Standardschrift
\renewcommand{\rmdefault}{ppl}


%%%-----------------------------------------------------------------------
\begin{document}


 	\begin{flushright}
 			{\small \sffamily{\textcolor{dunkelgrau.60}{Thesenpapier II f�r m�ndliche Pr�fung\\
 							\emph{Pascal Bernhard}\\
 							\emph{Prof. Dr. Manfred Kerner\\
 							Prof. Dr. Franz Walk}}}}
 	\end{flushright}


\vspace{0.8cm}

	\begin{center}
	\begin{framed}
	\sffamily{\textbf{{\Large �berregionale Zusammenarbeit in Europa\\
	am Beispiel der Euro-Regionen}}}
	\end{framed}
	\end{center} 	

\vspace{1.5cm}


\begin{quote}
\textit{ %
\textsf{\textbf{These:}\\
		Der unterschiedliche Handlungsspielraum der lokalen Selbstverwaltung, welcher durch das jeweilige nationale politische System gew�hrt wird, bestimmt Umfang und Erfolg transnationaler Kooperation auf regionaler Ebene.}}
\end{quote}


\vspace{0.8cm}


\subsection*{Regionale Zusammenarbeit in Europa}

\begin{itemize}
	\item \textbf{Antriebskr�fte regionaler Kooperation unterhalb der inter-gouvernementalen Ebene}


	\begin{compactitem}
		\item \textsl{Subsidiarit�tsprinzip:} regionale Probleme k�nnen besser von lokalen Institutionen bew�ltigt werden
		\item EU-Politik war nicht auf die Bed�rfnisse der territorialen Gliederung der Mitgliedsstaaten zugeschnitten
		\item bisherige M�glichkeiten der Zusammenarbeit auf europ�ischer Ebene nicht ausreichend
		\item durch Grenzen hervorgerufene Hindernisse sollen �berwunden werden
	\end{compactitem}


	\item \textbf{Rechtliche Grundlagen der grenz�berschreitenden Zusammenarbeit}

	\begin{compactitem}
		\item \textsl{Europ�ische Rahmenkonvention zur grenz�berschreitenden Zusammenarbeit (Konvention von Madrid, 1980)} gibt ersten rechtlichen Rahmen
		\item Zusatzprotokoll von 1995 gab erstmals die M�glichkeit, mit Gebietsk�rperschaften von Nicht-EU-Mitgliedern zusammenzuarbeiten
		\item \textsl{Vertrag von Maastricht 1992} erhob das Subsidiarit�tsprinzip zur allgemeing�ltigen Rechtsgrundlage
		\item Finanzierungm�glichkeiten f�r Projekte durch Europ�ische Territorial Zusammenarbeit (ETZ) nach Art. 158 EGV im Rahmen der Koh�sions- \& Strukturpolitik
	\end{compactitem}


	\item \textbf{Zusammenarbeit am Beispiel der Euro-Regionen}

	\begin{itemize}

		\item \textsl{Oberrheinregion (Frankreich - Schweiz - Deutschland)}


			\begin{compactitem}
			\item institutionalisierte Zusammenarbeit seit 1977 (Bonner Abkommen): Oberrheinrat und Oberrheinkonferenz
			\item Herausforderungen: Katastrophen- \& Umweltschutz, Raumpolitik (grenz�berschreitende Infrastruktur, Pendlerverkehr)m Tourismus
			\item weitreichender Handlungsspielraum in Deutschland und der Schweiz f�r lokale Selbstverwaltung
			\item durch zentralisistische Strukturen nur bedingt Kompetenzen f�r D\'{e}partements und Regionen in Frankreich
			\end{compactitem}


		\item \textsl{Euro-Region Niemen (Polen - Litauen - Russische F�deration)}


			\begin{compactitem}
			\item Zusammenarbeit auf den Gebieten grenz�berschreitende Infrastruktur \& Umweltschutz
			\item Kooperation zwischen Gebietsk�rperschaften der EU und einem Nicht-Mitglied (Russland)
			\item sehr unterschiedliche Kompetenzen f�r die Selbstverwaltungen zwischen Russland, Litauen und Polen
			\item kaum Erfahrung mit regionaler Autonomie
			\end{compactitem}


		\end{itemize}

\end{itemize}


\subsection*{Literatur}

\begin{compactitem}

	\item [\Rectsteel] \textbf{Europ�ische Union (1993):} Vertrag �ber die Europ�ische Union zusammen mit dem Wortlaut des Vertrages zur Gr�ndung der Europ�ischen Gemeinschaft (EUV), in: Amtsblatt der Europ�ischen Gemeinschaften C, 224/01, Br�ssel.


	\item [\Rectsteel] \textbf{Europarat (1980):} Europ�isches Rahmen�bereinkommen �ber die grenz�berschreitende Zusammenarbeit zwischen Gebietsk�rperschaften, Madrid, 21. Mai 1980.


	\item [\Rectsteel] \textbf{Euro-Region Niemen} \textsf{http://www.niemen.org.pl/} (Zugriff: 23.06.2014)

	\item [\Rectsteel] \textbf{Oberrheinkonferenz} \textsf{http://www.oberrheinkonferenz.org} (Zugriff: 25.06.2014)


	\item [\Rectsteel] \textbf{Sagan, Iwona (2012):} Polnische Regional- \& Metropolenpolitik, in: \textsl{Polen-Analysen}, Nr.103, S.2-6.


	\item [\Rectsteel] \textbf{Schneider, Eberhard (2002):} Staatliche Akteure russischer Au�enpolitik im Zentrum und in den Regionen, \textsl{SWP-Studie}, Stiftung Wissenschaft und Politik, Berlin.


	\item [\Rectsteel] \textbf{Sm\k{e}tkowski, Maciej (2013):} Regional Disparities In Central And Eastern European Countries: Trends, Drivers And Prospects, in: \textsl{Europe-Asia Studies}, Vol. 65 (8), S.1529-1554.


	\item [\Rectsteel] \textbf{Trinationale Metropolregion Oberrhein (2012):} Zivilgesellschaft in der Trinationalen Metropolregion Oberrhein - Synopse der B�rgerforen in Stra�burg, Karlsruhe und Basel, \textsf{http://www.rmtmo.eu/de/zivilgesellschaft.html?\\
	file=tl\_files/zivilgesellschaft-societe-civile/RMT-TMO-Buergerforen\%20am\\
	\%20Oberrhein\_SynopseDE.pdf} - (Zugriff: 27.06.2014)


	\item [\Rectsteel] \textbf{Yoder, Jennifer B. (2007):} Leading The Way To Regionalization In Post-Communist Europe: An Examination Of The Process And Outcomes Of Regional Reform In Poland, in: \textsl{East European Politics \& Society}, Vol. 21 (3), S.424-446.


	\item [\Rectsteel] \textbf{Ziemer, Klaus (2013):} Das politische System Polens, Wiesbaden, Bundesrepublik Deutschland.

\end{compactitem}


\end{document}