\documentclass[11pt,a4paper,ngerman]{scrreprt}
\usepackage[T1]{fontenc}
\usepackage{parskip}
\pagestyle{empty}
\usepackage[ansinew]{inputenc}


\usepackage{xcolor}
\usepackage[ngerman]{babel}
\usepackage{relsize}
\usepackage{paralist}
\usepackage{typearea}
\usepackage{setspace}
\usepackage{textcomp}
\usepackage{layouts}
\usepackage{amsmath}
\usepackage{fancybox}
\usepackage{marvosym}
\usepackage{pifont}

\definecolor{dunkelgrau.80}{gray}{0.20}
\definecolor{dunkelgrau.60}{gray}{0.40}
\definecolor{Dschungelgr\"{u}n}{cmyk}{0.99,0,0.52,0.2}
\definecolor{midblue}{rgb}{0.173,0.212,0.597}


\usepackage[top=10mm,bottom=10mm,marginparsep=8pt]{geometry}

%%%Palatino als Standardschrift
\renewcommand{\rmdefault}{ppl}


%%%-----------------------------------------------------------------------
\begin{document}




 	\begin{flushright}
 			{\small \sffamily{\textcolor{dunkelgrau.60}{Thesenpapier f�r m�ndliche Pr�fung\\
 							\emph{Pascal Bernhard}\\
 							\emph{Prof. Dr. Manfred Kerner\\
 							Prof. Dr. Franz Walk}}}}
 	\end{flushright}						

	

	\begin{center}
			{\shadowbox{\sffamily{\textbf{ 
				{\Large Der Ukraine-Russland--Konflikt}}}}}
	\end{center} 	

\vspace{1.5cm}


\begin{quote}
\textsf{\textbf{These I:}\\
		\textit{Erst durch das 'Opportunit�tsfenster' f�r politische Aktionen nach Sturz der Yanukovitsch-Regierung und der Destabilisierung der Situation durch den externen Akteur Russland, wurde die territoriale Einheit der Ukraine effektiv in Frage gestellt, obwohl grundlegende wirtschaftliche, politische und ethnische Probleme seit der Unabh�ngigkeit vorhanden sind.}}
\end{quote}


\vspace{0.8cm}


\subsection*{Ausgangssitution in der Ukraine}

\begin{itemize}
	\item \textbf{multi-ethnischer Staat ohne Tradition von Eigenstaatlichkeit, Demokratie und Marktwirtschaft}


	\begin{compactitem}
		\item Aufgabe des \textsl{Nation-Building}, jedoch kein nationaler Konsens zu Wirtschaftsmodell \& Gesellschaft
		\item hohe Erwartungen an Wohlstandsentwicklung
	\end{compactitem}


	\item \textbf{dysfunktionales politisches System}

	\begin{compactitem}
		\item Kompetenzstreitigkeiten zwischen Pr�sident und Parlament
		\item Ziele der Elite sind im Widerspruch mit den Erfordernissen des Transformationsprozesses
		\item undurchsichtige Privatisierungen, \textsl{State-Capture}, \textsl{Rent-Seeking} durch Oligarchen
		\item Interessen von Oligarchen und Machtk�mpfe zwischen ihnen bestimmen ukrainische Politik
	\end{compactitem}


	\item \textbf{kontinuierliche Wirtschaftskrise der Ukraine}

	\begin{compactitem}
	\item Absatzm�rkte f�r ukrainische Wirtschaft brachen weg (Landwirtschaft, Bergbau, R�stungsg�ter)
	\item Strukturwandel der sowjetischen Wirtschaft wurde von der Politik nicht vorangetrieben
	\item Lebensstandard der russischen Minderheit basierte auf sowjetischem System und war durch Marktreformen bedroht
	\item als einzige Branche florierte die Schattenwirtschaft
	\item sinkender Lebensstandard f�r die Mehrheit (Entlassungen, Inflation) bei zeitgleicher Bereicherung der Oligarchen
	\end{compactitem}

\end{itemize}


\pagebreak



\begin{quote}
\textsf{\textbf{These II:}\\
		\textit{Der Status quo ante der Ukraine wird voraussichtlich nicht wieder herzustellen sein, da weder die russische Minderheit, noch Russland hieran ein Interesse haben und dem Westen f�r entsprechende Schritte der Wille und die Einigkeit zu gemeinsamen Handeln fehlen.}}
\end{quote}

\vspace{0.8cm}


\begin{itemize}
	\item \textbf{Scheitern von Demokratie und Marktwirtschaft in der Ukraine}
	
	\begin{itemize}
	\item Staat Ukraine hat Erwartungen der Bev�lkerung entt�uscht\\
	\ding{225} kein Vertrauen in staatliche Institutionen\\
	\ding{225} der Staat Ukraine kann Erwartungen der 'Output-Legitimation' der russischen Minderheit nicht gerecht werden
	\end{itemize}


	\item \textbf{Pr�sident Vladimir Putins Gro�machtsambitionen}

	\begin{compactitem}
	\item Russland will alte Hegemonial-Stellung wiedererlangen
	\item Instabile Ukraine im Interesse russischer Au�enpolitik
	\item Ukrainische Demokratie als abschreckendes Beispiel f�r eigene Bev�lkerung
	\end{compactitem}


	\item \textbf{Unentschlossenheit des Westens (EU, NATO, USA)}

	\begin{compactitem}
	\item Garantieerkl�rungen Gro�britannien und der USA an die Ukraine nach Budapester Abkommen nicht eingehalten
	\item Uneinigkeit der EU-Mitglieder �ber Vorgehen gegen�ber Russland
	\item Interessenkonflikte zwischen Gesch�ftsbeziehungen mit Russland, Versorgung mir russischem Erdgas und dem Aufbau der Ukraine als Modell f�r Demokratie, Rechtsstaatlichkeit und Marktwirtschaft
	\item Ukraine als Mitglied f�r EU und NATO unattraktiv
	\end{compactitem}


\end{itemize}



\subsection*{Literatur}

{\small 


\begin{compactitem}



	\item [\Rectsteel] \textbf{Dev\v{c}i\'{c}, Jakov (2014):} Die Zeit in der Ukraine dr�ngt: Die Gewaltspirale muss gestoppt werden, in: \textsl{KAS L�nderberichte}, Mai 2014, Konrad-Adenauer--Stiftung e.V., Berlin.


	\item [\Rectsteel] \textbf{Engel, Simone (2008):} Das politische System in der Ukraine, in: \textsl{Aktuelle Ostinformationen}, Ausgabe 03-04, S.29-46.


	\item [\Rectsteel] \textbf{Gasimov, Zaur (2011):} Zwanzig Jahre sp�ter: (Ent-) Demokratisierung in den postsowjetischen R�umen, in: \textsl{Forum f�r osteurop�ische Ideen- und Zeitgeschichte}, 15. Jahrgang, Heft 1, S.159-188.


	\item [\Rectsteel] \textbf{Halbach, Uwe (2014):} Russland im Wertekampf gegen den "`Westen"', in: \textsl{SWP-Aktuell}, Nr. 43, Stiftung Wissenschaft und Politik, Berlin.


	\item [\Rectsteel] \textbf{Halling, Steffen \& Susan Stewart (2014):} Ukraine In Crisis: Challenges Of Developing I New Political Culture, in: \textsl{SWP Comments}, No.18, Stiftung Wissenschaft und Politik, Berlin.


	\item [\Rectsteel] \textbf{Korostelina, Carina (2003):} The Multi-Ethnic State-Building Dilemma: National And Ethnic Minorities' Identities In Crimea, in: \textsl{National Identities}, Vol.5 (2), S.141-159.


	\item [\Rectsteel] \textbf{Prusin, Alexander V. (2008):} The Land Between: Conflict in the East European Borderlands, Oxford, Oxfordshire, UK.


	\item [\Rectsteel] \textbf{Sasse, Gwendolyn (2001):} The 'New' Ukraine: A State Of Regions, in: \textsl{Regional \& Federal Studies}, Vol. 11 (3), S.69-100.


	\item [\Rectsteel] \textbf{Sasse, Gwendolyn (2002):} Conflict-Prevention In A Transition-State: The Crimean Issue In Post-Soviet Ukraine, in: \textsl{Politics}, Vol. 8 (2), S.1-26.


	\item [\Rectsteel] \textbf{Schr�der, Henning (2014):} Hat die Putin-Administration eine Strategie: Russische Innen- \& Au�enpolitik in der Ukraine-Krise, in: \textsl{Russland-Analysen}, Nr. 277, S.2-6.




\end{compactitem}
}
\end{document}
