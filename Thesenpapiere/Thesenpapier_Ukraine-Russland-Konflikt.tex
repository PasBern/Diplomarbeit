\documentclass[9pt,a4paper,ngerman]{report}
\usepackage[T1]{fontenc}
\usepackage{parskip}
\pagestyle{empty}
\usepackage[ansinew]{inputenc}


\usepackage{xcolor}
\usepackage[ngerman]{babel}
\usepackage{relsize}
\usepackage{paralist}
\usepackage{typearea}
\usepackage{setspace}
\usepackage{textcomp}
\usepackage{layouts}
\usepackage{amsmath}
\usepackage{fancybox}

\definecolor{dunkelgrau.80}{gray}{0.20}
\definecolor{dunkelgrau.60}{gray}{0.40}
\definecolor{Dschungelgr\"{u}n}{cmyk}{0.99,0,0.52,0.2}
\definecolor{midblue}{rgb}{0.173,0.212,0.597}


\usepackage[top=10mm,bottom=10mm,marginparsep=8pt]{geometry}



%%%-----------------------------------------------------------------------
\begin{document}



 
 	\begin{flushright}
 			{\small \sffamily{\textcolor{dunkelgrau.60}{Fragestellung\\
 							\emph{Pascal Bernhard}\\
 							Diplomarbeit\\
 							\emph{Prof. Dr. Manfred Kerner}}}}
 	\end{flushright}						

	

	\begin{center}
			{\shadowbox{\sffamily{\textbf{ 
				Energie im Baltikum
			\texttwelveudash{}Mehr Versorgungssicherheit durch gemeinsame europ\"{a}ische Energiepolitik?}}}}
	\end{center} 	

	
 
		
		\begin{itemize}
		
			\item{\textcolor{midblue}{\sffamily{\textbf{\emph{Kann die EU-Energiepolitik den baltischen L�nder aus ihrer Abh�ngigkeit von russischen Gaslieferungen helfen?}}}}}
 
 			\begin{itemize}
 			
 				\item{Anwendung von IB-Theorien \& Integrationsans\"{a}tzen 
 							auf den europ��ischen Integrationsprozess}
 				\item{\textsl{aktuelle politische Relevanz}}
 				
 					\begin{itemize}
 					
 					\item{\textcolor{dunkelgrau.60}{Energieversorgungssicherheit 
 								als gesamteurop\"{a}ische Herausforderung}}
 					\item{\textcolor{dunkelgrau.60}{Gasstreitigkeiten zwischen 
 								Russland und der 
 								Ukraine zeigen, dass Handlungsbedarf besteht}}
 					\end{itemize}
 					
 			\end{itemize}
 
		\end{itemize} 
 
 		\begin{itemize}
 			
 			\item{\textcolor{midblue}{\textbf{\textsl{Theoretische Grundlagen}}}}
 				
 				\begin{itemize}
 					
 					\item{(liberaler) Intergouvernementalismus zur Erkl\"{a}rung der 
 								Integrationsentscheidungen in der EU}
 					\item{Mitgliedsl\"{a}nder sind die Herren der Vertr\"{a}ge
 								\texttwelveudash {} Opt-out--M\"{o}glichkeiten im EGV}
 					\item{Energiesektor ist eng mit dem Nationalstaat verbunden}
 						
 						\begin{itemize}
 							
 							\item{\textcolor{dunkelgrau.60}{Energie ist essentiell 
 											f\"{u}r	Wirtschaft und Gesellschaft eines Landes}}
 							\item{\textcolor{dunkelgrau.60}{historisch gesehen waren 
 										Energieunternehmen stets unter staatlicher 
 										Kontrolle}}
 										
 						\end{itemize}
 	
 	
 	\reversemarginpar
 						
 					\item{\textsl{Intergouvernementalismus} erkl\"{a}rt bisherige
 								Integrationsschritte nur unzul\"{a}nglich}
 					\item{aufgrund der spezifischen Charakteristika des Energiesektors macht es f\"{u}r die EU-Mitglieder Sinn, energiepolitische Fragen supranational zu regeln}
 	
 	\normalmarginpar
 							
 							\begin{itemize}
 							
 								\item{\textcolor{dunkelgrau.60}{\"{A}berwindung des Kooperationsdilemmas (Abwehr einer 		\textsl{divide-et-impera} \texttwelveudash {}  
								Strategie der F\"{o}rderl\"{a}nder)}}
 											
 								\item{\textcolor{dunkelgrau.60}{Effizienz\"{u}berlegungen}}
 								
 							\end{itemize}
 						
 					\item{Institutionengef\"{u}ge der EU gibt jedoch auch anderen 
 								Akteuren ein Mitspracherecht}
 					\item{der Rational-Choice \texttwelveudash {} Ansatz bietet sich an f\"{u}r die Analyse der Entscheidungsprozesse (Principal-Agent \texttwelveudash {} Struktur, Verhandlungsmodelle)}
 		
 			\end{itemize}
 			
 		\end{itemize}
 		
 \begin{center}	
 		\textbf{\textsl{Die Energiepolitik auf europ\"{a}ischer Ebene kommt nicht durch intergouvernementale Entscheidungen der EU-Mitglieder zu Stande, sondern entsteht durch Verhandlung zwischen den Institutionen Kommission, Rat und EP, deren Integrationspr\"{a}ferenzen ber\"{u}cksichtigt werden m\"{u}ssen}}
 	\end{center}
 							
	\begin{itemize}
		
		
		\item{\textcolor{midblue}{\textbf{\textsl{abh\"{a}ngige Variable}}}}	
			
			\begin{itemize}
				\item{Grad der Integration in Bezug auf Energiepolitik}
				\item{\textsl{Operationalisierung (ordinales Messniveau					\texttwelveudash {} 7-stufige Skala):}}
							
					\begin{itemize}
						\item{\textcolor{dunkelgrau.60}{\"{U}bertragung von 
									Kompetenzen an europ\"{a}ische Organe}}
						\item{\textcolor{dunkelgrau.60}{Bewilligung von Mitteln zur 
									Fortentwicklung einer europ. Energiepolitik}}
						\item{\textcolor{dunkelgrau.60}{politische Beschl\"{u}sse zur 
									Koordination nationaler Politiken}}
					\end{itemize}
					
			\end{itemize}
			
		\item{\textcolor{midblue}{\textbf{\textsl{unabh\"{a}ngige Variablen}}}}
		
			\begin{itemize}
				\item{Pr\"{a}ferenzen der Mitgliedsstaaten zu Integration}

				\item{Vorschl\"{a}ge (Pr\"{a}ferenzen) der Kommission}

				\item{Rolle des Europ\"{a}ischen Parlaments}

				\item{\textsl{Operationalisierung (ordinales Messniveau	\texttwelveudash {} 7-stufige Skala):}}


		
									
				\begin{itemize}
							
				
					
					\item{\textcolor{dunkelgrau.60}{Gr\"{u}nbuch und White Papers, Statements, Entw\"{u}rfe des DG TREN, Staff Working Documents}}
												
					\item{\textcolor{dunkelgrau.60}{ Stellungnahmen des	Rates,  Protokolle (soweit zug\"{a}nglich), COREPER und Energiekommittee}}				
					\item{\textcolor{dunkelgrau.60}{Fortschrittsberichte der Kommission}}
								
					\item{\textcolor{dunkelgrau.60}{Position des EP}}
								
				\end{itemize}
							
			\end{itemize}
		
		\item{\textcolor{midblue}{\textbf{\textsl{Operationalisierung anhand von Beispielf�llen}}}}

		      \begin{itemize}

			\item{\textbf{Beispiel: prim�rer Energietr�ger Erdgas}}

			\item{\textsl{Vernetzung mit Pipelines aus Westeuropa}}



			    \begin{itemize}

			      \item{Versorgung bei Engp�ssen oder Unterbrechung der Lieferungen aus Russland}

			      \item{Potential von Schiefergas (Shale-Gas) aus Polen}

			      \item{politisches Verh�ltnis zu Polen, insbesondere im Falle Litauens}

			    \end{itemize}
			  

			\item{\textsl{Bau eines Terminals f�r Fl�ssiggas}}
		
	   
			    \begin{itemize}

			      \item{Frage nach Standort des Terminals}

			      \item{Finanzierung des Projektes \texttwelveudash {} Hilfen durch EU-Mittel?}

			      \item{Welche F�rderl�nder als Quellen des verschifften Fl�ssiggases stehen bereit? \texttwelveudash {} Sind ausreichend F�rderkapazit�ten in Quatar oder Trinidad\&Tobago, bzw. Algerien vorhanden?}

			      \item{Stellt das polnische Fl�ssiggasterminal eine Konkurrenz f�r das baltische Projekt dar?}

			      \item{K�nnen die baltischen L�nder ihre Differenzen �ber das Projekt regeln? Erleichtern europ�ische Institutionen die L�sung der Probleme?}

			    \end{itemize}
			   
		      \item{\textbf{Beispiel: sekund�rer Energietr�ger Elektrizit�t}}


		      \item{\textsl{Schwedischer Vorschlag einer Vernetzung der Ostseeanrainerstaaten}}

			  \begin{itemize}
			    \item{Kosten des Projektes \texttwelveudash {} Hilfen durch EU-Mittel?}

			     \item{K�nnen die baltischen L�nder ihre Differenzen �ber das Projekt regeln? Erleichtern europ�ische Institutionen die L�sung der Probleme?}

			     \item{Welche Auswirkungen hat die von der Kommission gew�nschte Liberalisierung des europ�ischen Strommarktes, insbesondere der Entflechtung der Bereiche Erzeugung und �bertragung von Strom f�r das Ostseeprojekt?}

			     \item{Stehen die Interessen der gro�en Mitgliedsstaaten einer EU-Regelung entgegen, welche dieses Projekt beg�nstigen w�rde?}

			  \end{itemize}
			  

			\item{Welches Potential haben regenerative Energiequellen f�r die Stromerzeugung in den baltischen Staaten? Wie stark kann hier�ber die Abh�ngigkeit von Energieimporten aus dem Ausland, insbesondere Russland gesenkt werden? (Strom \& Heizenergie)}

		    \end{itemize}




		
	\end{itemize} 							
 							
\end{document}
