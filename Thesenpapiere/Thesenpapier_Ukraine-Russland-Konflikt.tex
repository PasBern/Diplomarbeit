\documentclass[9pt,a4paper,ngerman]{scrreprt}
\usepackage[T1]{fontenc}
\usepackage{parskip}
\pagestyle{empty}
\usepackage[ansinew]{inputenc}


\usepackage{xcolor}
\usepackage[ngerman]{babel}
\usepackage{relsize}
\usepackage{paralist}
\usepackage{typearea}
\usepackage{setspace}
\usepackage{textcomp}
\usepackage{layouts}
\usepackage{amsmath}
\usepackage{fancybox}

\definecolor{dunkelgrau.80}{gray}{0.20}
\definecolor{dunkelgrau.60}{gray}{0.40}
\definecolor{Dschungelgr\"{u}n}{cmyk}{0.99,0,0.52,0.2}
\definecolor{midblue}{rgb}{0.173,0.212,0.597}


\usepackage[top=10mm,bottom=10mm,marginparsep=8pt]{geometry}

%%%Palatino als Standardschrift
\renewcommand{\rmdefault}{ppl}


%%%-----------------------------------------------------------------------
\begin{document}




 	\begin{flushright}
 			{\small \sffamily{\textcolor{dunkelgrau.60}{Thesenpapier f�r m�ndliche Pr�fung\\
 							\emph{Pascal Bernhard}\\
 							\emph{Prof. Dr. Manfred Kerner\\
 							Prof. Dr. Franz Walk}}}}
 	\end{flushright}						

	

	\begin{center}
			{\shadowbox{\sffamily{\textbf{ 
				{\Large Der Ukraine-Russland--Konflikt}}}}}
	\end{center} 	

\vspace{1.5cm}


\begin{quote}
\textit{ %
\textsf{\textbf{These I:}\\
		Erst durch das 'Opportunit�tsfenster' f�r politische Aktionen nach Sturz der Yanukovitsch-Regierung und der Destabilisierung der Situation durch den externen Akteur Russland, wurde die territoriale Einheit der Ukraine effektiv in Frage gestellt, obwohl grundlegende wirtschaftliche, politische und ethnische Probleme seit der Unabh�ngigkeit vorhanden sind.}}
\end{quote}


\vspace{0.8cm}


\subsection*{Ausgangssitution in der Ukraine}

\begin{itemize}
	\item \textbf{multi-ethnischer Staat ohne Tradition von Eigenstaatlichkeit, Demokratie und Marktwirtschaft}


	\begin{compactitem}
		\item Aufgabe des \textsl{Nation-Building}, jedoch kein nationaler Konsens zu Wirtschaftsmodell \& Gesellschaft
		\item hohe Erwartungen an Wohlstandsentwicklung
	\end{compactitem}


	\item \textbf{dysfunktionales politisches System}

	\begin{compactitem}
		\item Kompetenzstreitigkeiten zwischen Pr�sident und Parlament
		\item Ziele der Elite im Widerspruch mit den Erfordernissen des Transformationsprozesses
		\item Politik und Justiz f�r Partikularinteressen von Oligarchen missbraucht
		\item undurchsichtige Privatisierungen, \textsl{State-Capture}, \textsl{Rent-Seeking} durch Oligarchen
	\end{compactitem}


	\item \textbf{kontinuierliche Wirtschaftskrise der Ukraine}

	\begin{compactitem}
	\item Absatzm�rkte f�r ukrainische Wirtschaft brachen weg (Landwirtschaft, Bergbau, R�stungsg�ter)
	\item Strukturwandel der sowjetischen Wirtschaft wurde von der Politik nicht vorangetrieben
	\item Lebensstandard der russischen Minderheit basierte auf sowjetischem System und war durch Marktreformen bedroht
	\item als einzige Branche florierte die Schattenwirtschaft
	\item sinkender Lebensstandard f�r die Mehrheit (Entlassungen, Inflation) bei zeitgleicher Bereicherung der Oligarchen
	\end{compactitem}

\end{itemize}

\vspace{0.8cm}



\begin{quote}
\textit{ %
\textsf{\textbf{These II:}\\
		Der Status quo ante der Ukraine wird voraussichtlich nicht wieder herzustellen sein, da weder die russische Minderheit, noch Russland hieran ein Interesse haben und dem Westen f�r entsprechende Schritte der Wille und die Einigkeit zu gemeinsamen Handeln fehlen.}}
\end{quote}


\subsection*{Literatur}


	\item [\Rectsteel] \textbf{Aalto, Pami \& Kirsten Westphal (2008):} Introduction, in: \textsl{Aalto, Pami (Hrsg.)}: The EU-Russian 	Energy Dialogue: Europe's Future Energy Security, Aldershot, Hampshire, UK, S.1-22.


	\item [\Rectsteel] \textbf{Aalto, Pami \& Kirsten Westphal (2008):} Introduction, in: \textsl{Aalto, Pami (Hrsg.)}: The Land Between: Conflict in the East European Borderlands, Aldershot, Oxfordshire, UK.

\end{document}
